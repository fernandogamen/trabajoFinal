\documentclass[10pt,twocolumn,draft]{article} 
\usepackage{times}
\usepackage{graphicx}
\usepackage{amssymb}
\usepackage{url,hyperref}
\usepackage{cite}


\begin{document}

\title{\textbf{Implementaci\'on de Risk con algoritmo de Minimax.}}
\author{Prieto, Estefan\'ia$^{[1]}$\\
Galicia, Fernando$^{[1]}$\\
Galv\'an, Antonio$^{[1]}$\\
$^{[1]}$Facultad de Ciencias, UNAM.\\
\\
prietaslastiene@fciencias.unam.mx\\
gagay@fciencias.unam.mx\\
agalvan@fciencias.unam.mx\\
}

\twocolumn[\begin{@twocolumnfalse}
\maketitle
\thispagestyle{empty}
\begin{abstract}
El universo de \textbf{RISK} resulta ser demasiado amplio de tal forma que este proyecto busca una implementaci\'on sencilla basada en el algoritmo de \textit{Minimax} nativa de la teor\'ia de juegos.
\\
Para buscar \'esta implementaci\'on se ha decidido acotar el universo de tal forma que no contar\'a con las cartas presentes en el juego usual, reducir la cantidad de dados y tambi\'en el territorio de expansi\'on.
\\
La direcci\'on que se le busca dar a este proyecto es el poder modelar de forma acotada el juego del \textbf{RISK} con teor\'ia de juegos.
\\
\end{abstract}
\end{@twocolumnfalse}]


\section{Introducci\'on.}

Hacer ac\'a la introducci\'on de minimax.


\section{Juegos con informaci\'on perfecta. [MODIFICAR A INFORMARCI\'ON SEMI-PERFECTA Y ARGUMENTAR EL POR QUE PODEMOS ADAPTARLO AS\'I.]}
Explicar por que catalogamos al \textbf{RISK} c\'omo un juego de informaci\'on perfecta y por que hemos elegido esta implementaci\'on.


% % % % % % % % % % % % % % % % % % % % % % % % % % % % % % % % % % % % % % % % % % % % % % % %
% % % % % % % % % % % % RISK ACOTADO % % % % % % % % % % % % % % % % % % % % % % % % % % % % % %
% % % % % % % % % % % % % % % % % % % % % % % % % % % % % % % % % % % % % % % % % % % % % % % % 
\section{Risk acotado.}

Tal y c\'omo se plantea en el juego original (\textbf{\textit{ve\'ase \cite{RISK}}}) el objetivo del juego continua siendo la dominaci\'on total de un territorio dado, de tal forma
que el juego queda concluido cu\'ando todos los territorios quedan bajo la dominaci\'on de 
un jugador.\\
En esta implementaci\'on acotaremos la cantidad de continentes, es decir, el desarrollo sera unicamente en un solo continente, tambi\'en la cantidad de dados se ve acotada a unicamente dos dados y restringido a dos jugadores.\\

Sin embargo mantendremos las dem\'as condiciones iniciales con respecto a las tropas y al equivalente de tropas en cada territorio, es decir:
\begin{list}{*}{}
\item Cada unidad representa una \textit{Armada}.
\item Cada \textit{Caballer\'ia} representa 5 unidades.
\item Cada \textit{Artiller\'ia} representa 10 unidades.
\end{list}

Teniendo ya esto definido, entonces, cada jugador tendr\'a un ejercito inicial de [...]

\section{Descripci\'on del agente.}
Aqu\'i es donde describimos el comportamiento del agente.


\section{Funci\'on de evaluaci\'on e implementaci\'on.}
Describir aqu\'i los detalles de la implementaci\'on.

\section{Especificaciones del programa.}
Aqu\'i van las especificaciones de c\'omo es que programamos nuestro \textbf{RISK}, lenguaje, resultados, etc.

\section{Conclusiones.}
Informar las conclusiones que hemos encontrado en nuestra implementaci\'on.


% %Toda la bibliografía consultada debe de estar anexada en el archivo
% %referencias.bib 
\newpage
\bibliographystyle{plain}	
\bibliography{referencias.bib}{}





\end{document}